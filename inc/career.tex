\section{职业分析}
\subsection{头脑风暴}
\frame
{	\frametitle{\secname}
	\begin{block}{\subsecname}
		\centering
		\begin{tikzpicture}
		[	scale = .38, 
			transform shape
		]
			\tikzstyle{every node} = [font = \large]
			\path
			[	mindmap, 
				concept color = black, 
				text = white
			]
			node[concept]{职业选择}[clockwise from = 0]
			child
			[	grow = 0, 
				concept color = black! 25! red
			]
			{	node[concept]{社会形势}[clockwise from = 0]
				child[grow = -60]{node[concept]{高端制造}}
				child[grow = 0]{node[concept]{航空航天}}
				child[grow = 60]{node[concept]{智能、控制、新材料……}}
			}
			child
			[	grow = 120, 
				concept color = black! 25! green
			]
			{	node[concept]{个人兴趣}[clockwise from = 0]
				child[grow = 60]{node[concept]{控制}}
				child[grow = 120]{node[concept]{智能}}
				child[grow = 180]{node[concept]{航电}}
			}
			child
			[	grow = 240, 
				concept color = DeepSkyBlue						
			]
			{	node[concept]{行业现状}[clockwise from = 0]
				child[grow = 180]{node[concept]{能力不足}}
				child[grow = 240]{node[concept]{经验需要积累}}
				child[grow = 300]{node[concept]{人才饱和,需要时间熬}}
			}; 
		\end{tikzpicture}
	\end{block}
}
\subsection{SWOT分析}
\frame
{	\frametitle{\secname}
	\begin{block}{\subsecname}
		\centering
	\end{block}
}
\note
{	优势:目标明确,兴趣所在,专业符合;
	\\劣势:知识储备,社会因素;
	\\机会:学校支持,有相关平台;
	\\威胁:起步较晚,行业不成熟。
}