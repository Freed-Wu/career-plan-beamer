\section{执行计划}
\subsection{短期计划}
\frame
{	\frametitle{\secname}
	\begin{block}{\subsecname}
		\begin{tikzpicture}
			\duck
			[	cap = red! 80! black, 
				football = white! 85! yellow, 
				jacket = red, 
				buttons = brown! 50! black
			]
		\end{tikzpicture}
		\begin{flushright}
			\begin{tikzpicture}
				\duck
				[	graduate = gray! 20! black, 
					tassel = red! 70! black, 
					book = \scalebox{0.6}{\tiny 毕业证}, 
					bookcolour = blue! 50! black, 
					jacket = black! 50! gray, 
					speech = {毕业啦!}, 
					bubblecolour = cyan! 20! white
				]
			\end{tikzpicture}
		\end{flushright}
	\end{block}
}
\note
{	优秀的完成本科学历,掌握扎实的专业知识。
	\\在本科阶段学习相关知识,为以后工作打下基础。
	\\积极参加科研比赛,提高科研能力与团队协作能力。
	\\考上研究生,在研究生阶段获得更高的知识水平。
}
\subsection{长期计划}
\frame
{	\frametitle{\secname}
	\begin{block}{\subsecname}
		\begin{tikzpicture}
			\duck
			[	tshirt = lightgray, 
				jacket = blue! 50! black, 
				tie = blue! 80! black, 
				squareglasses = blue! 50! black, 
				shorthair, 
				signpost = \scalebox{0.4}{\parbox{2cm}{\color{DeepSkyBlue}\centering Science\\first}}, 
				signcolour = brown! 70! gray, 
				signback = white! 80! brown
			]
		\end{tikzpicture}
		\begin{flushright}
			\begin{tikzpicture}
				\duck
				[	recedinghair = gray, 
					beard = gray, 
					glasses = red! 50! black, 
					crozier = brown! 80! black, 
					tshirt = red! 50! black, 
					think = {emm...}, 
					bubblecolour = white! 95! yellow
				]
			\end{tikzpicture}
		\end{flushright}
	\end{block}
}
\note
{	本科生阶段:在学好本专业知识的前提下,自学和本专业相关的技能。
	\\多与同学合作参加比赛,提高团队协作能力。
	\\阅读课外书籍,提升个人修养。
	\\研究生阶段:涉足更难的领域,大幅提高自己科研能力。
	\\对本专业有更深的理解,掌握实验设备的使用。
	\\参加工作阶段:在工作过程中不断积累经验,最终具备系统设计、技术开发的能力。
}
