\makeatletter
%-------------------------------------------------------------------------------------------------------
\title{职业生涯规划}
\subtitle{}
\author[吴振宇]{}
\renewcommand\insertauthor
{	\newline\makebox[3em][s]{学生}:\makebox[3em][s]{吴振宇}
	\\\makebox[3em][s]{辅导员}:\makebox[3em][s]{苏欣}
}
\institute[南京理工大学]{\includegraphics[width = 0.4\linewidth]{NJUST.ai}}
\date{\today}
%-------------------------------------------------------------------------------------------------------
\usepackage{pgfpages}
\setbeameroption{show notes on second screen = right}
\renewcommand\pgfsetupphysicalpagesizes
{	\pdfpagewidth\pgfphysicalwidth
	\pdfpageheight\pgfphysicalheight
}
%\setbeamercovered{highly dynamic}
%-------------------------------------------------------------------------------------------------------
\usecolortheme{whale}
\usecolortheme{orchid}
\CenterWallPaper{1}{wood.jpg}
\setbeamercolor{background canvas}{bg=}
\setbeamercolor{frametitle}
{	bg = brown, 
	fg = white
}
\setbeamercolor{title}
{	fg = white, 
	bg = brown
}
\setbeamercolor{block title}{bg = brown}
\setbeamercolor{block title example}
{	use = {normal text, example text}, 
	fg = example text.fg! 75! normal text.fg, 
	bg = brown text.bg! 75! black
}
\setbeamercolor{fine separation line}{}
\setbeamercolor{item projected}{fg = black}
\setbeamercolor{palette sidebar primary}
{	use = normal text, 
	fg = normal text.fg
}
\setbeamercolor{palette sidebar quaternary}
{	use = structure, 
	fg = structure.fg
}
\setbeamercolor{palette sidebar secondary}
{	use = structure, 
	fg = structure.fg
}
\setbeamercolor{palette sidebar tertiary}
{	use = normal text, 
	fg = normal text.fg
}
\setbeamercolor{section in sidebar}{fg = brown}
\setbeamercolor{section in sidebar shaded}{fg = grey}
\setbeamercolor{separation line}{}
\setbeamercolor{sidebar}{bg = brown}
\setbeamercolor{sidebar}{parent = palette primary}
\setbeamercolor{structure}{fg = brown}
\setbeamercolor{subsection in sidebar}{fg = brown}
\setbeamercolor{subsection in sidebar shaded}{fg = grey}
%-------------------------------------------------------------------------------------------------------
\setbeamerfont{frametitle}{size = \large}
\setbeamerfont{footnote}{size = \tiny}
\setbeamerfont{subsection in toc}{size = \footnotesize}
\setbeamerfont{caption}{size = \scriptsize}
%-------------------------------------------------------------------------------------------------------
\useinnertheme[shadow]{rounded}
%\useinnertheme{circles}
%\useinnertheme{rectangles}
%\useinnertheme{inmargin}
\setbeamertemplate{caption}[numbered]
%-------------------------------------------------------------------------------------------------------
%\useoutertheme{infolines}
%\useoutertheme{split}
%\useoutertheme[subsection = true]{smoothbars}
%\useoutertheme{smoothtree} 
%\useoutertheme[hook]{tree}
\useoutertheme{shadow}
\useoutertheme[footline = authorinstitutetitle]{miniframes}
\setbeamertemplate{navigation symbols}{}
\setbeamertemplate{footline}
{	\begin{beamercolorbox}
	[	ht = 2.5ex, 
		dp = 1.125ex, 
		leftskip = .3cm, 
		rightskip = .3cm plus1fil
	]{author in head/foot}
		\leavevmode
		\usebeamerfont{author in head/foot}
		\usebeamercolor[fg]{author in head/foot}
		\insertshortauthor
		\hfill
		\usebeamerfont{institute in head/foot}
		\usebeamercolor[fg]{institute in head/foot}
		\insertshortinstitute
	\end{beamercolorbox}
	\begin{beamercolorbox}
	[	ht = 2.5ex, 
		dp = 1.125ex, 
		leftskip = .3cm, 
		rightskip = .3cm plus1fil
	]{title in head/foot}
		\usebeamerfont{title in head/foot}
		\usebeamercolor[fg]{title in head/foot}
		\insertshorttitle
		\hfill
		\usebeamerfont{frame number}
		\usebeamercolor[fg]{frame number}
		第\insertframenumber 页~共\inserttotalframenumber 页
	\end{beamercolorbox}
}
%-------------------------------------------------------------------------------------------------------
\AtBeginSubsection
{	\frame<handout:0>
	{	\frametitle{目录}
		\tableofcontents[current, currentsubsection]
	}
	\addtocounter{framenumber}{-1}
}
%-------------------------------------------------------------------------------------------------------
\makeatother
